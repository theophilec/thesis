\newcommand{\sujet}[1]{\renewcommand{\sujet}{#1}}
\newcommand{\auteur}[1]{\renewcommand{\auteur}{#1}}
\newcommand{\encadrant}[1]{\renewcommand{\encadrant}{#1}}

\newcommand{\objectif}[1]{%
\vspace{.1cm}
\begin{singlespace}
\tikzstyle{titlebox}=[rectangle,inner sep=10pt,inner ysep=10pt,draw=curcolor,draw]%
\tikzstyle{title}=[fill=white]%
\bigskip\noindent\begin{tikzpicture}
\node[titlebox] (box){%
    \begin{minipage}{0.88\textwidth}
#1
    \end{minipage}
};
\node[title] at (box.north west) {\color{curcolor}  Key takeaways};
\end{tikzpicture}\bigskip%
\vspace{.1cm}
\minitoc
\end{singlespace}
\newpage
}

\newcommand{\myparagraph}[1]{\paragraph{#1}\mbox{} \vskip .5\baselineskip \par}

\newcommand{\T}[1]{T\textsubscript{#1}}


%%% Macro pour la mise en forme de la liste des publiations
\newcommand{\Myprod}[3]{ % sans DOI
\begin{singlespace}{#1}. "\textit{{#2}}". {#3}.
\end{singlespace}}

\newcommand{\MyprodwithDOI}[4]{ % avec DOI
\begin{singlespace}{#1}. "\textit{{#2}}". {#3}.\newline
{DOI: \href{https://doi.org/#4}{#4}}
\end{singlespace}}

% Chapitre sans mise en forme particulière (remerciements, conclusion
\newcommand{\addchapnonumber}[1]{
\phantomsection
\addtocounter{chapter}{1}
\chapter*{#1}
\addcontentsline{toc}{chapter}{#1}
\markboth{#1}{#1}
\setcounter{section}{0}
}

%%% Macro pour mise en forme de l'abstract et résumé avec mots clés
\newcommand{\AddResumeAbstract}{
\chapter{Résumé}

\thefrabstract
\vskip 2em \noindent\makebox[\linewidth]{\rule{.5\linewidth}{0.4pt}}
\vfill
\noindent \textbf{Mots clés :}
\thefrkeywords

\chapter{Abstract}

\theenabstract
\vskip 2em \noindent\makebox[\linewidth]{\rule{.5\linewidth}{0.4pt}}
\vfill
\noindent\textbf{Keywords :}
\theenkeywords
}

\newcommand{\fix}{\marginpar{FIX}}
\newcommand{\new}{\marginpar{NEW}}


\newcommand{\Nystrom}[1]{{Nystr\"om}}

\providecommand{\abs}[1]{\lvert{#1}\rvert}
\providecommand{\set}[1]{\{#1\}}
\providecommand{\scal}[2]{\left\langle{#1},{#2}\right\rangle}
\providecommand{\nor}[1]{\bigl\|{#1}\bigr\|}
\providecommand{\ran}[1]{\operatorname{ran}{#1}}
\providecommand{\tr}{\operatorname{Tr}}

% Numbers
%\newcommand{\cal}[1]{\mathcal{#1}}
\newcommand{\R}{\mathbb R}
\newcommand{\CC}{\mathbb C}
\newcommand{\N}{\mathbb N}
\newcommand{\BH}{{{\mathcal B}(\hh)}}

% Hilbert spaces
\newcommand{\hh}{\mathcal H}
\newcommand{\kk}{\mathcal K}

% Greek letters
\newcommand{\la}{\lambda}
\newcommand{\eps}{\epsilon}
\newcommand{\rank}[1]{\operatorname{rank}\left(#1\right)}
\newcommand{\diag}[1]{\operatorname{diag}\left(#1\right)}
\newcommand{\lspan}[1]{\operatorname{span}\{#1\}}
\newcommand{\lspanc}[2]{\overline{\operatorname{span}\{#1~|~#2\}}}
\newcommand{\supp}[1]{\operatorname{supp}{#1}}
\newcommand{\argmin}[1]{\mathop{\operatorname{argmin}}_{#1}}

\newcommand{\prob}[1]{{\mathbb P}[#1]}
\newcommand{\expect}[1]{{\mathbb E}[#1]}

\newcommand{\G}{{\cal G}}
\newcommand{\Ln}{{L^2(\X,\rho_n)}}

\renewcommand{\set}[2]{{\{{#1}~|~{#2}\}}}
\newcommand{\Z}{{\cal Z}}
\newcommand{\X}{{\cal X}}
\newcommand{\Y}{{\cal Y}}
\newcommand{\vz}{{\mathbf z}}
\newcommand{\vx}{{\mathbf x}}
\newcommand{\vm}{\tilde{\mathbf m}}

\newcommand{\EE}{{\mathcal E}}
\newcommand{\cc}{C}
\newcommand{\rhox}{{\rho_{\X}}}
\newcommand{\Ltwo}{{L^2(\X,\rhox)}}
\renewcommand{\S}{{S}}
\renewcommand{\L}{{L}}
\newcommand{\C}{{C}}
\renewcommand{\k}{{K}}
\newcommand{\K}{K_n}
\newcommand{\Cl}{\C_\la}
\newcommand{\Cn}{{\C}_n}
\newcommand{\Cnl}{{\C}_{n\lambda}}
\newcommand{\Cj}{{\C}_j}
\newcommand{\Cjl}{{\C}_{j\lambda}}
\newcommand{\Aj}{A_j}
\newcommand{\Bj}{B_j}
\newcommand{\tj}{\theta_j}
\newcommand{\Vj}{V_j}
\newcommand{\Cm}{{\C}_m}
\newcommand{\Sn}{S_n}
\newcommand{\Sj}{S_j}
\newcommand{\yj}{\bar{y}_j}
\newcommand{\fh}{f_\hh}
\newcommand{\Tj}{T_j}
\newcommand{\Tjl}{T_{j\la}}
\newcommand{\Dj}{D_j}
\newcommand{\Djl}{D_{j\la}}
\newcommand{\Zm}{Z_m}
\newcommand{\Pm}{P_m}
\newcommand{\yn}{\widehat y_n}

\newcommand{\fn}{f_n}
\newcommand{\frho}{f_\rho}

\newcommand{\vy}{{\mathbf y}}
\newcommand{\vc}{{\mathbf c}}

\newcommand{\dd}{\mathrm{d}}

\newcommand{\MM}{{\bf M}}

\newcommand{\eqals}[1]{\begin{align*}#1\end{align*}}
\newcommand{\eqal}[1]{\begin{align}#1\end{align}}
\newcommand{\bpr}{\begin{proof}}
\newcommand{\epr}{\end{proof}}
\newcommand{\be}{\begin{equation}}
\newcommand{\ee}{\end{equation}}

\newcommand{\bi}{\begin{itemize}}
\newcommand{\ei}{\end{itemize}}

\newcommand{\bex}{\begin{example}}
\newcommand{\eex}{\end{example}}
\newcommand{\Lt}{L^2(\X, \rhox)}
\newcommand{\Ltn}{L^2(\X, \hat \rho _\X)}
\newcommand{\lc}{\left\{}
\newcommand{\rc}{\right\}}
\newcommand{\lp}{\left(}
\newcommand{\rp}{\right)}



\providecommand{\norh}[1]{\left\Vert{#1}\right\Vert_\hh}
\providecommand{\norf}[1]{\left\Vert{#1}\right\Vert_\mathcal F}
\providecommand{\norop}[1]{\lVert{#1}\rVert_{op}}
\providecommand{\scalh}[2]{\left\langle{#1},{#2}\right\rangle_\hh}

\newcommand{\mixing}{\sigma}
\DeclareMathOperator{\partialpsd}{\textsc{PartialEval}}
\DeclareMathOperator{\productpsd}{\textsc{Product}}
\DeclareMathOperator{\marginalpsd}{\textsc{Marginal}}
\DeclareMathOperator{\integralpsd}{\textsc{Integral}}
\DeclareMathOperator{\filtersteppsd}{\textsc{FilterStep}}
\newcommand\independent{\protect\mathpalette{\protect\independenT}{\perp}}
\def\independenT#1#2{\mathrel{\rlap{$#1#2$}\mkern2mu{#1#2}}}
\newcommand{\barhat}[1]{\tilde{{#1}}}

\usepackage[ruled]{algorithm2e}
