\section{Markov kernels \& Hidden Markov Models}


\subsection{Tools and notation}
Let $(E, \mathcal E)$ be a measurable space. We denote $\mathcal M(E)$ the set of finite signed measures on $(E, \mathcal E)$, $\mathcal M_+(E)$ the set of finite positive measures, $\mathcal M_0(E)$ the set of finite signed measures which sum to $0$ and $\mathcal P(E)$ the set of probability distributions on $(E, \mathcal E)$. Let $(F, \mathcal F)$ be a second measurable space.

\subsection{Total variation}

\begin{definition}[Total variation norm]
Let $\xi$ be a finite signed measure on $(E, \mathcal E)$. The total variation of $\xi$ is $\Vert \xi \Vert_{TV}= \xi^+(E) + \xi^-(X)$ where $\xi^+, \xi^-$ is the Jordan-Hahn decomposition of $\xi$. If $\xi$ admits a density with respect to the Lebesgue measure, then $\Vert \xi \Vert_{TV} = \int \vert \xi(x)\vert dx$.
\end{definition}
\begin{proposition} Let $\mu$ and $\nu$ be two finite measures on $(E, \mathcal E)$. Then,
\begin{align}
    \Vert \bar \mu - \bar \nu \Vert = \frac{\Vert \mu - \nu \Vert}{\mu(E)} + \frac{\vert \mu(E) - \nu(E)\vert}{\mu(E)}
\end{align}
In particular,
\begin{align}
    \Vert \bar \mu - \bar \nu \Vert \leq \frac{2\Vert \mu - \nu \Vert}{\mu(E)}
\end{align}
\end{proposition}
\subsection{Transition kernels}
We report the essential results relative to transition kernels taken \cite{cappehmm}.
\begin{definition}[Transition kernel]\label{def:transition-kernel}
A function $Q:E \times \mathcal F \to \mathbb R^+$ is an unnormalized transition kernel if:
\begin{itemize}
    \item for all $x\in E$, $Q(x, \cdot)$ is a positive measure on $(F, \mathcal F)$;
    \item for all $A\in \mathcal F$, $x\mapsto Q(x, A)$ is measurable.
\end{itemize}
$Q$ is normalized if for any $x\in E$, $Q(x, F)=1$. When $E=F$ and $Q$ is normalized, $Q$ is said to be a Markov transition kernel.

By abuse of notation, when $Q$ admits a density with respect to the Lebesgue measure, we denote it $Q$ as well, i.e. $Q(x, dy)=Q(x, y)dy$.
\end{definition}

Note that $R_n(u, x)= Q(u, x)G(x, z_n)$ is indeed an unnormalized transition kernel.

\begin{definition}[Effects of kernels]
    Let $K$ be an unnformalized kernel on $(E, \mathcal E)\times (F, \mathcal F)$, $\mu\in\mathcal M_+(E)$ and $f$ a bounded function $f:E\to\mathbb R$.

    Then, $K\mu\in\mathcal M_+(F)$ with for any $A\in\mathcal F$,
    \begin{align}
        K\mu(A) = \int K(u, A)\mu(du),
    \end{align}

    and $Kf: E \to \mathbb R$ is a bounded function with for any $u\in E$,
    \begin{align}
        Kf(u) = \int K(u, A)\mu(du).
    \end{align}
\end{definition}
\subsection{Hidden Markov Models}
\begin{definition}[Hidden Markov Model]
Let $(E, \mathcal E)$ and $(F, \mathcal F)$ be two measurable spaces. Let $Q$ and $G$ denote a Markov transition kernel on $(E, \mathcal E)$ and $G$ denote a transition kernel from $(E, \mathcal E)$ to $(F, \mathcal F)$. Let $T$ be the Markov transition kernel defined on the product space $(E \times F, \mathcal E \otimes \mathcal F)$ by
\begin{equation}
    \forall (x, y) \in E\times F, \forall C\in\mathcal E \otimes\mathcal F, ~T\left[(x, y), C\right] = \int \int 1_C((x^\prime, y^\prime)) Q(x, dx^\prime) G(x^\prime, dy^\prime)
\end{equation}

The Markov Chain $\lbrace X_k, Y_k \rbrace_{k\geq 0}$ with Markov transition kernel $T$ and initial distribution $\nu \otimes G$, where $\nu$ is a probability distribution on $(E, \mathcal E)$ is called a \textbf{Hidden Markov Model}.

We denote $\mathbb P_\nu$ and $\mathbb E_\nu$ the probability measure and corresponding expectation associated with the process $\lbrace (X_k, Y_k)\rbrace$ over $\left((E\times F)^{\mathbb N}, (\mathcal E \otimes \mathcal F)^{\otimes \mathbb N}\right)$.
\end{definition}

Throughout this work, we assume that for any $x\in E$, $Q(x, \cdot) \ll \lambda(\cdot)$ and $G(x, \cdot) \ll \lambda(\cdot)$ where $\lambda$ is the Lebesgue measure over $(E, \mathcal E)$, and we denote $Q(x, dx^\prime) = Q(x, x^\prime)dx^\prime$ and $G(x, dy)= G(x, y)dy$.


\begin{definition}[Filtering distribution]
Let $\nu$ be a probability distribution over $(E, \mathcal E)$ and $n \geq 0$. We denote $\pi_n^\nu$ the conditional distribution of $X_n$ given $Y_{1:n}$, i.e.
\begin{itemize}
    \item $\pi_n^\nu$ is a transition kernel from $F^n$ to $E$
    \item $\pi_n^\nu$ satisfies for any bounded function $f: E \to \mathbb R$,
    \begin{equation}
        \mathbb E_\nu \left[ f(X_n) \vert Y_{1:n}\right] = \int f(x)\pi^\nu_n(Y_{1:n}, dx)
    \end{equation}
\end{itemize}
\end{definition}

\subsection{Mixing kernels}
\begin{definition}[Mixing kernel]
    We say a kernel $K(x, dy)$ is mixing if there exists a positive constant $\mixing > 0$ and a non-negative measure $\xi$ such that for any $x\in E$ and $A\in\mathcal F$,
    \begin{equation}
        \mixing\xi(A) \leq K(x, A) \leq \frac{1}{\mixing}\xi(A).
    \end{equation}
\end{definition}
If $K$ is mixing for a measure $\xi$ and a constant $\mixing$ we write that $K$ is $\mixing$-$\xi$-mixing.

\begin{remark}
    Note that we can add the constraint that $\xi$ be normalized. Indeed, if $K$ is $\mixing$-$\xi$-mixing with $\xi(E)\neq 1$, then $K$ if $\bar \mixing$-$\bar\xi$-mixing with $\bar\mixing = \mixing \times \min\left(\xi(E), \frac{1}{\xi(E)}\right)\leq \mixing$.
\end{remark}

\begin{proposition}[Sufficient condition for mixing when $K$ admits a density]
    If $K(x, dy) = \kappa(x, y)dy$ and there exists $\mixing >0$ and a measure density $\xi$ such that for any $x\in E$ and $y\in F$,
    \begin{equation}
        \mixing\xi(y) \leq \kappa(x, y) \leq \frac{1}{\mixing}\xi(y).
    \end{equation}

    then $K$ is mixing with constant $\mixing$ and $\xi(A)=\int 1_A(y)\xi(y)dy$.
\end{proposition}

\begin{proposition}
    If $K$ is mixing with $\mixing$ and $\xi$, then for any $\mu$,
    \begin{equation}
        \mixing\xi(A)\leq K\mu(A)\leq \frac{1}{\mixing}\xi(A).
    \end{equation}
\end{proposition}
\begin{proof}
    \begin{equation}
    \mixing\xi(A)\leq K\mu(A)=\int 1_A(x)\int K(u, dx)\mu(du) \leq \int 1_A(x)\int \frac{1}{\mixing}\xi(dx)\mu(du)=\frac{1}{\mixing}\xi(A)
    \end{equation}
\end{proof}


\subsection{Optimal kernel $R_n$}\label{sec:optimal_kernel_appendix}

\begin{definition}
Let $y_n \in F$.
Let $R_n : E \times \mathcal E \to \mathbb R^+$ a kernel defined by:
    for any bounded function $f\in\mathcal F_b(E)$, and $u\in E$,
    \begin{align}
        R_n(u, f) = \int f(x)Q(u, x)G(x, y_n)dx
    \end{align}
We call $R_n$ the optimal kernel.
\end{definition}
For alternative definitions see for example \citealp[page 220]{cappehmm}.


\subsection{Hilbert metric}

\begin{definition}[Comparable measures] Let $\mu$ and $\nu$ be two measures on $(E, \mathcal E)$. $\mu$ and $\nu$ are said to be comparable if there exists $\infty > a, b >0$ such that for any $A\in \mathcal E$,
\begin{equation}
    a\nu(A) \leq \mu(A) \leq b\nu(A)
\end{equation}
\end{definition}

\begin{proposition}
Let $\mu$ and $\nu$ be two comparable measures on $(E, \mathcal E)$ and let $K$ be an unnormalized transition kernel on $(E, \mathcal E)$. Then, for any $n\geq 0$, $K^n\mu$ and $K^n\nu$ are comparable.
\end{proposition}

\begin{proof} By recursion,
    \begin{equation}
        bK\nu(A) \leq K\mu(A) = \int 1_A(x) \int K(u, dx)\mu(du) \leq aK\nu(A)
    \end{equation}
\end{proof}

\begin{definition}[Hilbert metric]
   Let $\mu$ and $\nu$ be two comparable probability distributions on $(E, \mathcal E)$. The Hilbert metric between $\mu$ and $\nu$ is defined as
\begin{align}
    h(\mu, \nu) = \log\left[\frac{\sup_{A\in\mathcal E, \nu(A)>0} \frac{\mu(A)}{\nu(A)}}{\inf_{A\in \mathcal E, \nu(A) > 0}\frac{\mu(A)}{\nu(A)}}\right]
\end{align}
\end{definition}

\begin{proposition} If $\mu$ and $\nu$ are two comparable probability distributions on $(E, \mathcal E)$ and furthermore they both admit densities then,
\begin{equation}
h(\mu, \nu)=\log\left[\frac{\textrm{ess}\sup_x \frac{\mu(x)}{\nu(x)}}{\textrm{ess}\inf_x\frac{\mu(x)}{\nu(x)}}\right] = \log \left[\left\Vert\frac{\mu(x)}{\nu(x)}\right\Vert_\infty\left\Vert\frac{\nu(x)}{\mu(x)}\right\Vert_\infty\right]
\end{equation}
\end{proposition}

\begin{definition}[Birkhoff contraction coefficient]
Let $K$ be an unnormalized transition kernel. Define $\tau(K)$ such that
\begin{equation}
    \tau(K) = \sup_{0 < h(\mu, \nu) < \infty} \frac{h(K\mu, K\nu)}{h(\mu, \nu)}
\end{equation}
where the supremum is taken over comparable, positive measures $\mu$ and $\nu$.
\end{definition}

\begin{proposition}[Properties of the Birkhoff coefficient]\label{prop:hilbert}
Let $K$ be an unnormalized transition kernel.
\begin{itemize}
    \item $\tau$ is sub-multiplicative, i.e. $\tau(KL)\leq \tau(K)\tau(L)$
    \item $\tau \leq 1$
    \item if in addition $K$ is $\mixing$-$\xi$-mixing, then $\tau(K) \leq \frac{1- \mixing^2}{1 + \mixing^2}$
\end{itemize}
\end{proposition}

These properties are proven in \cite{cohen}, which studies the Hilbert metric is detail.

\begin{proof} We have $\tau(K) = \tanh\left[\frac{1}{4}\Delta(K)\right]$ where $\Delta(K) = \sup_{\mu, \mu^\prime}h(K\mu, K\mu^\prime)$ ($\Delta$ for diameter). Since $K$ is $\mixing$-$\xi$-mixing, if $\mu$ and $\nu$ are two finite measures on $(E, \mathcal E)$,
\begin{equation}
    \mixing^2K\mu(A)\leq\mixing\xi(A)\leq K\mu(A)\leq \frac{1}{\mixing}\xi(A) \leq \frac{1}{\mixing^2}K\mu^\prime(A)
\end{equation}
   So, $h(K) \leq \log\left(\frac{1}{\mixing^4}\right)\leq \frac{1 - \mixing^2}{1 + \mixing^2}$.
\end{proof}


\begin{proposition}[Total variation - Hilbert comparaisons]\label{prop:comparaison}
Without any hypotheses on $\mu, \nu$ finite measures, if $\bar \mu = \mu / \mu(E)$ and $\bar \nu = \nu / \nu(E)$ are normalized counterparts to $\mu$ and $\nu$, then
    \begin{equation}\label{eq:tv-to-hilbert}
        \Vert \bar\mu - \bar\nu \Vert \leq \frac{2}{\log(3)}h(\mu, \nu)
    \end{equation}
If in addition, $K$ is an $\mixing$-mixing kernel,
\begin{equation}\label{eq:hilbert-to-tv}
    h(K\mu, K\nu) \leq \frac{1}{\mixing^2} \Vert \mu - \nu \Vert
\end{equation}
These inequalities are proven in \cite{cohen}.
\end{proposition}
\section{Proof of \cref{theorem:learning}}\label{sec:proof-learning}
%TODO THC 16/07
\subparagraph{Notation and background results} In this section, $\Vert \cdot \Vert$ without any subscript denotes $\Vert \cdot\Vert_{W_2^\beta(\Omega)}$. Let $f$ a target function defined on $\Omega = (-1, 1)^d$ that verifies \cref{assumption:target_function}, and let $g = \sqrt{f}$. Let $\nu = \min(1, d/2\beta)$. Let $\epsilon > 0$ and $\delta > 0$. In this proof all constants $C$ and exponents $\alpha, \gamma, \rho, \ldots$ are independent of $f, g, \hat g, \hat f$ unless otherwise stated. Only $\beta, \nu$, $\tilde \nu$ and $\theta$ have importance. We recall also the Gagliardo-Nirenberg inequality, that will be used later.
\begin{lemma}[Gagliardo-Nirenberg inequality\cite{wendland2004scattered}]\label{lemma:gargliano}
Let $u\in L^\infty(\mathbb R^d)\cap W^{m, 2}(\mathbb R^d)$ then
    \begin{align}
        \Vert u \Vert_{L^\infty(\Omega)} \leq C \Vert u\Vert_{W^{m, 2}(\Omega)}^\theta \Vert u\Vert_{L^2(\Omega)}^{1-\theta}
    \end{align}
    where $\theta = \frac{d}{m}$ and $C$ is independent of $u$.
\end{lemma}


\subparagraph{Setting the function space $\mathcal H_\eta$ and $g_{\tau, \epsilon}$}
Set $\tau = \epsilon^{-2/\beta}$ and $\lambda = \epsilon^{\frac{2\beta + d}{\beta}}$ and $\mathcal H_\eta$ the reproducing kernel Hilbert space associated to $k_\eta$ where $\eta = \tau 1_d$.

\subparagraph{Existence and properties of $g_{\tau, \epsilon}$}
As a consequence of the Stein extension theorem (see Corollary A.3 of \cite{rudi2021psd}), there exists a function $\tilde g \in W^\beta_2(\Omega)$ such that $\tilde g_{\vert \Omega}=g$ and $\Vert \tilde g\Vert_{W_2^\beta(\Omega)} \leq\Vert g\Vert_{W_2^\beta(\Omega)}$ and $\Vert \tilde g\Vert_{L^\infty(\Omega)}\leq C\Vert g\Vert_{L^\infty(\Omega)}$.
%
According to \cite{rudi2021psd} and \cite{sampling-ulysse} (Proposition 7) there exists $g_{\tau, \epsilon}\in\mathcal H_\eta$ and some constants $C_1, C_2$ depending only on $\beta, d$ and independent of $g$ such that:
\begin{align}
&\Vert g - g_\tau \Vert_{L^\infty(\Omega)} \leq C_1 \epsilon^{1- \nu}\Vert g\Vert_{W^\beta_2(\Omega)}\label{eq:g-g-tau},\\
&\Vert g_\tau \Vert_{\mathcal H_\eta}\leq C_2 \Vert g\Vert_{W^\beta_2(\Omega)}\epsilon^{-d/2\beta}\label{eq:g-tau}.
\end{align}

\subparagraph{Learning $\hat g$ and $\hat f$} Let $M, n \in \N$, draw $X\in\mathbb R^{n \times d}$ the training set and $\tilde X \in \mathbb R^{M \times d}$ the set of anchor points. Define $y \in \mathbb R^n$ such that $y_i = g(x_i)$ for any $1 \leq i \leq n$ and $x_i$ is the $i$-th row of $X$.
%
We formalize \cref{eq:learning-problem} explicitly as a kernel ridge regression problem below:
%
\begin{align}\label{eq:learning-problem-appendix}
    \min_{a\in\mathbb R^d} \frac{1}{n} \vert a^\top \Phi_{\eta}(X_i) - y_i\vert^2 + \lambda a^\top K a
\end{align}
%
where $\Phi_\eta(x) = (k_\eta(x, \tilde x_1) \dots k_\eta(x, \tilde x_M))^\top \in\mathbb R^M$ and $K\in\mathbb R^{M\times M}$ is given by $K_{ij}=k_\eta(\tilde x_i, \tilde x_j)$.
%
Problem \cref{eq:learning-problem-appendix} is strongly convex and has a unique solution $\hat a$.
%
We denote $\hat g$ the Gaussian Linear Model defined by $\hat a, \eta,$ and $\tilde X$, i.e. such that for any $x \in\mathbb R^d$, $\hat g(x) = \hat a^\top \Phi_\eta(x)$. Define $\hat f$ the Gaussian PSD Model $\hat f= \hat g^2$ where $\hat f(x) = \Phi_\eta(x)^\top \hat a\hat a^\top \Phi_\eta(x)$.
%
The analysis of \cite{sampling-ulysse} shows that when $M \geq C^\prime \log^d(\frac{1}{\epsilon})\log(\frac{1}{\delta\epsilon})\epsilon^{-d/\beta}$, and $n \geq C^\prime \epsilon^{-d/\beta}\log \frac{1}{\delta}$, then there exist two constants $C_3$ and $C_4$ independent of $g$ and $\hat g$ such that $\hat g$ verifies the following inequalities each with probability at least $1-\delta$,
\begin{align}
   &\Vert \hat g\Vert_\mathcal H\leq C_3\Vert g\Vert_{W^\beta_2(\Omega)}\epsilon^{-d/2\beta}\label{eq:hat-g}\\
    &\Vert g - \hat g\Vert_{L^2(\Omega)} \leq C_4 \Vert g\Vert_{W^\beta_2(\Omega)}\epsilon.\label{eq:g-hat-g}
\end{align}

\subparagraph{Deriving the bound for $g - \hat{g}$ in $L^\infty(\Omega)$}
In particular, using the triangle inequality and combining \cref{eq:g-g-tau} and \cref{eq:g-hat-g}, there exists a constant $C_5$ such that with probability at least $1-\delta$,
\begin{align}
    &\Vert g_{\tau, \epsilon} - \hat g\Vert_{L^2(\Omega)} \leq C_5\Vert g\Vert_{W^\beta_2(\Omega)}\epsilon.\label{eq:g-tau-hat-g}
\end{align}
%
We now have all the ingredients to bound $\Vert g - \hat g\Vert_{L^\infty(\Omega)}$ in high probability. First, notice that :
%
\begin{align}
   \Vert g - \hat g\Vert_{L^\infty(\Omega)} \leq \Vert g - g_{\tau, \epsilon}\Vert_{L^\infty(\Omega)}+ \Vert g_{\tau, \epsilon} - \hat g\Vert_{L^\infty(\Omega)}.
\end{align}
%
We apply the Gagliardo-Nirenberg inequality (\cref{lemma:gargliano}) to $\Vert g_{\tau, \epsilon} - \hat g\Vert_{L^\infty(\Omega)}$:
\begin{align}
\Vert \hat g - g_{\tau, \epsilon}\Vert_{L^\infty(\Omega)}\leq C \Vert \hat g - g_{\tau, \epsilon}\Vert_{W^{m}_2(\Omega)}^\theta \Vert \hat g - g_{\tau, \epsilon}\Vert_{L^2(\Omega)}^{1-\theta}.
\end{align}
with $\theta = \frac{d}{2m}$ and $m \geq d/2$ (we fix $m$ when we optimize the exponents below) and $C$ is a constant independent of $\hat g$ and $g_{\tau, \epsilon}$.

The Sobolev norm $\Vert h \Vert_{W^m_2}(\mathbb R^d)$ is upper bounded by the rkhs norm $\Vert h \Vert_{\mathcal H_\eta}$ for $h\in \mathcal H_\eta \subset W^m_2(\mathbb R^d)$. Thus, applying the triangle inequality and bounds \cref{eq:g-tau,eq:hat-g,eq:g-tau-hat-g} there exists a constant $C_6$  such that with probability at least $1 - 2\delta$,
\begin{equation}\label{eq:g-hat-g-tau-infty}
\Vert \hat g - g_{\tau, \epsilon}\Vert_{L^\infty(\mathbb R^d)}\leq C_6\Vert g\Vert\epsilon^{1 - \theta - \theta d/2\beta}.
\end{equation}

Combining \cref{eq:g-g-tau,eq:g-hat-g-tau-infty}, there exist two constants $C, C^\prime > 0$ such that with probability at least $1 - 2\delta$,
\begin{align}
   \Vert g- \hat g\Vert_{L^\infty(\Omega)} \leq C\Vert g\Vert_{W^\beta_2(\Omega)} \epsilon^{1-\nu} + C^\prime\Vert g\Vert_{W^\beta_2(\Omega)}\epsilon^{1- \theta - \theta d/2\beta}
\end{align}

Choosing $m = \beta + \frac{d}{2}$, there exists a constant $C_7>0$  such that for $\epsilon$ small enough with probability at least $1 - 2\delta$,
\begin{align}
    \Vert g - \hat g\Vert_{L^{\infty}(\Omega)}\leq C\Vert g\Vert_{W^\beta_2(\Omega)}\epsilon^{1- d/2\beta}.
\end{align}

\subparagraph{Bounding $\Vert g+ \hat g\Vert_{L^\infty(\Omega)}$}
Using the triangle inequality and \cref{eq:hat-g}, there exists a constant $C_8>0$ and $\rho > 0$ such that with probability at least $1- \delta$,
\begin{align}
    \Vert g +\hat g\Vert_{L^\infty(\Omega)} \leq 2\Vert g\Vert_{L^\infty(\Omega)} + \Vert g-\hat g\Vert_{L^\infty(\Omega)} \leq 2\Vert g\Vert_{L^\infty(\Omega)} + C\Vert g\Vert_{W^\beta_2(\Omega)}\epsilon^{1- d/2\beta}.
\end{align}

\subparagraph{Bounding $\Vert f - \hat f\Vert_\infty$}
Notice that since $a^2 - b^2 = (a-b)(a+b)$,
\begin{align}
    \Vert \hat f - f \Vert_{L^\infty(\Omega)} \leq \Vert g - \hat g\Vert_{L^\infty(\Omega)} \Vert g +\hat g\Vert_{L^\infty(\Omega)}.
\end{align}

Combining the above bounds:

\begin{align}
    \Vert \hat f - f \Vert_{L^\infty(\Omega)} \leq C\Vert g\Vert^2_{W^\beta_2(\Omega)}\epsilon^{2- d/\beta} + C^\prime \Vert g\Vert^2_{W^\beta_2(\Omega)}\epsilon^{1 - d/2\beta}
\end{align}

Since $\beta > d/2$, under the conditions on $M, n$, there exists a constant $C^{\prime\prime}= C + C^\prime$ depending only on $\Omega, d, \beta$ and independent of $f, g,\hat g, \hat f$ such that with probability at least $1-3\delta$,

\begin{align}
    \Vert \hat f - f \Vert_{L^\infty(\Omega)} \leq C^{\prime\prime}\Vert g\Vert^2_{W^\beta_2(\Omega)} \epsilon^{1 - d/2\beta}
\end{align}


\section{Proof of \cref{theorem:bound-diagonal}}\label{sec:proof-bound}
In this section $E = (-1, 1)^d$ and $\Omega = E \times F$. Without loss of generality, we assume that $F=E$. Of course, any compact can be considered.

\subsection{Propagation of one-step errors}

We generalize the proof technique in \cite{oudjane} to take into consideration general sequences of densities.

\begin{proposition}
Let $\pi_0\in\mathcal P(E)$. Let $(z_k)_{k\geq 1}$ and $(\pi_k)_{k\geq 1}$ the optimal filter sequence computed on the $(z_k)_{k\geq 1}$ and initialized at $\pi_0$. Let $(\mu_k)\in\mathcal P(E)^\mathbb N$ a sequence of distributions such that $\mu_0 = \pi_0$. Then, for any $n\geq 0$:
\begin{align}\label{eq:telescopic}
    \mu_n - \pi_n = \sum_{k=1}^n \bar R_{n:k+1}(\mu_k) - \bar R_{n:k}(\mu_{k-1})
\end{align}
with the notation that $\bar R_{n:k} = \bar R_n \circ \dots \circ \bar R_k$ and $R_{n+1:n}=id$ if $k > l$.
\end{proposition}
\begin{proof}
    Telescopic sum:
\begin{align}
    \mu_n - \pi_n &= \sum_{k=1}^n \bar R_{n:k+1}(\mu_k) - \bar R_{n:k}(\mu_{k-1})\\
    &= \mu_n - \bar R_n(\mu_{n-1}) + \bar R_{n}(\mu_{n-1} - \bar R_{n:n-1}(\mu_{n-2}) \ldots + \bar R_{n:2}(\mu_1) - \underbrace{\bar R_{n:1}(\mu_0)}_{=\bar R_{n:1}(\pi_0)=\pi_n}
\end{align}
\end{proof}



\begin{proposition}[Optimal filter stability]\label{prop:optimal-forgetting}
Let $(\pi_n^\nu)$ and $(\pi_n^\mu)$ two sequences of optimal filters initialized at $\nu$ and $\mu$ respectively, and computed on the same data sequence $z_1, \ldots, z_n$. We assume that for all $n\geq 1$, $R_n$ verifies \cref{assumption:mixing}. Then,
\begin{align}
    \Vert \pi^\nu_n - \pi^\mu_n \Vert \leq \frac{2}{\mixing^2\log 3}\left(\frac{1 - \mixing^2}{1 + \mixing^2}\right)^{n-1}\Vert \mu - \nu \Vert_{TV}
\end{align}
\end{proposition}

\begin{proof}
   By \cref{eq:tv-to-hilbert},
\begin{align}
    \Vert \pi^\mu_n - \pi^\nu_n \Vert_{TV} & \leq h(R_{n:1}(\mu_n), R_{n:1}(\nu_n))
\end{align}
Since $R_{n}, \ldots, R_1$ are mixing, $R_{1}\mu$ and $R_1\nu$ are comparable and we can apply \cref{prop:hilbert} with the Hilbert contraction coefficient $\tau(R_{n:2})\leq \tau_\mixing^{n-1}\leq\left(\frac{1-\mixing^2}{1+\mixing^2}\right)^{n-1}$.
\begin{align}
    \Vert \pi^\mu_n - \pi^\nu_n \Vert_{TV} & \leq \frac{2}{\log 3}\tau_\mixing^{n-1}h(R_{1}(\mu), R_{1}(\nu))
\end{align}
And finally, applying \cref{eq:hilbert-to-tv},
\begin{align}
    \Vert \pi^\mu_n - \pi^\nu_n \Vert_{TV} & \leq \frac{2}{\mixing^2\log(3)}\tau_\mixing^{n-1}\Vert \mu - \nu\Vert_{TV}
\end{align}
\end{proof}
\begin{proposition}\label{prop:lemma_bound}
    Let $E = (-1, 1)^d$. Let $\mixing>0$ and $\xi\in\mathcal P(E)$ such that the optimal kernel $R_n:E\times E\to \mathbb R^+$ is $\mixing$-$\xi$ mixing for all $n$. Let $\mu_n$ a sequence of positive, finite measures on $E$ such that $\mu_0=\pi_0$. Then,
\begin{align}
    \Vert \mu_n - \pi_n \Vert_{TV} & \leq \delta_n + \frac{2}{\log 3}\frac{1}{\mixing^2}\sum_{k=1}^{n-1}\tau^{n-k-1}\delta_k
\end{align}
where $\delta_n = \Vert \mu_n - \bar R_n(\mu_{n-1})\Vert_{TV}$.
\end{proposition}

\begin{proof}
Applying the triangle inequality with the total variation norm to \cref{eq:telescopic},
\begin{align}
    \Vert \mu_n - \pi_n \Vert_{TV} &\leq \sum_{k=1}^n \Vert \bar R_{n:k+1}(\mu_k) - \bar R_{n:k}(\mu_{k-1}) \Vert_{TV} & \\
    & \leq \Vert \mu_n - \bar R_n(\mu_{n-1})\Vert_{TV} +\sum_{k=1}^{n-1} \Vert \bar R_{n:k+1}(\mu_k) - \bar R_{n:k}(\mu_{k-1}) \Vert_{TV} &
\end{align}
We apply \cref{eq:tv-to-hilbert} from \cref{prop:comparaison},
\begin{align}
    \Vert \mu_n - \pi_n \Vert_{TV}& \leq \Vert \mu_n - \bar R_n(\mu_{n-1})\Vert_{TV} + \frac{2}{\log 3}\sum_{k=1}^{n-1}h(R_{n:k+1}(\mu_k), R_{n:k}(\mu_{k-1}))
\end{align}
Since $R_{k+1}$ and $R_k$ are mixing, $R_{k+1}\mu_k$ and $R_{k+1}R_{k}\mu_{k-1}$ are comparable and we can apply \cref{prop:hilbert} with the Hilbert contraction coefficient $\tau(R_{n:k+2})\leq \tau_\mixing^{n-k-1}= \left(\frac{1-\mixing^2}{1+\mixing^2}\right)^{n-k-1}$.
\begin{align}
\Vert \mu_n - \pi_n \Vert_{TV}& \leq \Vert \mu_n - \bar R_n(\mu_{n-1})\Vert + \frac{2}{\log 3}\sum_{k=1}^{n-1}\tau_\mixing^{n-k-1}h(R_{k+1}\mu_k, R_{k+1}R_k\mu_{k-1})
\end{align}
Since $R_{k+1}$ is mixing, we apply \cref{eq:hilbert-to-tv} from \cref{prop:comparaison}:
\begin{align}
\Vert \mu_n - \pi_n \Vert_{TV}& \leq \Vert \mu_n - \bar R_n(\mu_{n-1})\Vert + \frac{2}{\mixing^2\log 3}\sum_{k=1}^{n-1}\tau_\mixing^{n-k-1}\frac{1}{\mixing^2}\Vert \mu_k - \bar R_{k}(\mu_{k-1})\Vert_{TV}
\end{align}
where we use that the Hilbert contraction coefficient is sub-multiplicative and can be bounded away from $1$ as a function of $\mixing$ (we denote it $\tau_\mixing$ this upper-bound given in \cref{prop:hilbert}). By denoting $\delta_n = \Vert \bar\mu_n - \bar{R}_n(\bar\mu_{n-1})\Vert_{TV}$ we obtain the result.
\end{proof}

We now prove \cref{lemma:bound}.

\begin{proof}
We apply the triangle inequality to $\Vert \pi^\nu_n - \hat\pi_0\Vert_{TV}$:
\begin{align}
    \Vert \pi^\nu_n - \hat\pi_0\Vert_{TV} \leq  \Vert \pi^\nu_n - \pi_n^{\hat \pi_0}\Vert_{TV} + \Vert \pi^{\hat\pi_0}_n - \hat\pi_n\Vert_{TV}
\end{align} and the result follows from \cref{prop:optimal-forgetting} and \cref{prop:lemma_bound}.
\end{proof}

\subsection{Bound with smooth, bounded approximations}

Let $\hat Q : E \times E\to\mathbb R^+$ and $\hat G: E \times F \to \mathbb R^+$ two bounded approximations of $Q$ and $G$. Given a sequence $(z_k)\in F^\mathbb N$, define the approximate non-negative kernel $\hat R_n(u, x) = \hat Q(u, x)\hat G(x, z_n)$ defined on $E\times E$. We introduce $(\hat\pi_n)$ the sequence of probability distributions computed using the recursion $\hat \pi_{n} = \hat R_n\hat\pi_{n-1}/R_n\hat\pi_{n-1}(E)=\barhat{R}_n(\hat\pi_{n-1})$.

\begin{proposition}\label{prop:bound-delta-n}
Let $\delta_n = \Vert \barhat{R}_n(\hat\pi_{n-1}) - \bar R_n(\hat\pi_{n-1})\Vert_{TV}$. Then,
\begin{align}
    \delta_n \leq & \frac{2}{\mixing\xi(E)}\left(\Vert G-\hat G\Vert_{L^\infty(E\times E)} + C_d\Vert \hat G\Vert_{L^\infty(E\times E)}\Vert Q-\hat Q\Vert_{L^\infty(E\times E)}\right)
\end{align}
where $C_d$ is a constant independent of $Q, G,\hat G, \hat Q$.
\end{proposition}

\begin{proof}
$\delta_n$ is the total variation distance between two probability distributions. Recall that
\begin{align}
    \Vert \bar \mu - \bar \nu\Vert_{TV} \leq \frac{2}{\mu(E)}\Vert \mu - \nu\Vert_{TV}
\end{align}
for any $\mu, \nu$ positive, finite measures on $E$ and $\bar \mu, \bar \nu$ their normalized counterparts.
Thus,
\begin{align}
    \delta_n & \leq \frac{2}{R_n\hat\pi(E)}\Vert\hat R_n(\hat\pi_{n-1}) - R_n(\hat\pi_{n-1})\Vert_{TV}.
\end{align}

Recall that for any $\mu\in\mathcal P(E)$, $\mixing\xi(E) \leq R_n\mu(E)\leq \frac{1}{\mixing}\xi(E)$, which allows us to control the denominator:
\begin{align}
    \delta_n & \leq \frac{2}{\mixing\xi(E)}\Vert\hat R_n(\hat\pi_{n-1}) - R_n(\hat\pi_{n-1})\Vert_{TV}.
\end{align}

Because we will be able to bound the quality of approximation between $Q$ and $\hat Q$ (and between $G$ and $\hat G$), we split the above expression:
\begin{align}
\Vert\hat R_n(\hat\pi_{n-1}) - R_n(\hat\pi_{n-1})\Vert_{TV} \leq& \int\int \hat \pi_{n-1}(u)\vert Q(u, x)G(x, z_n) - \hat Q(u, x)\hat G(x, z_n)\vert dudx\\
& \leq \underbrace{\int\int Q(u, x) \hat \pi_{n-1}(u)\vert G(x, z_n) - \hat G(x, z_n)\vert dudx}_{A_n}\label{eq:delta-a} \\
& + \underbrace{\int\int \hat G(u, x) \hat \pi_{n-1}(u)\vert Q(u, x) - \hat Q(u, x)\vert dudx}_{B_n}.\label{eq:delta-b}
\end{align}

First, let us bound $A_n$.
\begin{align}
A_n &= \int\int Q(u, x) \hat \pi_{n-1}(u)\vert G(x, y_n) - \hat G(x, y_n)\vert dudx\\
&\leq \Vert G-\hat G\Vert_{L^\infty(E\times E)}\int \hat\pi_{n-1}(u)\int Q(u,x)dxdu\\
&= \Vert G-\hat G\Vert_{L^\infty(E\times E)}.
\end{align}
where we used that $Q$ is a transition kernel (i.e. that $Q(u, E)=1$ for all $u\in E$) and that $\hat\pi_{n-1}$ is a distribution.

Second, let us bound $B_n$:
\begin{align}
B_n &= \int\int \hat G(x, z_n) \hat \pi_{n-1}(u)\vert Q(u, x) - \hat Q(u, x)\vert dudx\\
    &\leq \Vert Q-\hat Q\Vert_{L^\infty(E\times E)}\int \hat \pi_{n-1}(u)du \int \hat G(x, z_n)dx\\
    &\leq \textrm{Vol}(\Omega)\Vert \hat G\Vert_{L^\infty(E\times E)}\Vert Q-\hat Q\Vert_{L^\infty(E\times E)}
\end{align}
where we again used that $\hat \pi_{n-1}$ is a probability distribution.
\end{proof}

\subsection{Putting everything together}

We assume that $E=F$ without loss of generality (simply replace $D = 2d$ for $Q$ and $D=d+d^\prime$ for $G$). We choose $\beta > D/2$. Thus, here $\Omega = E\times E\subset \mathbb R^{D}$ where $D=2d$.

Let $\gamma > 0$ and apply \cref{theorem:learning} with its parameter $\epsilon = \gamma$, Then we have that there exist $\hat G$ and $\hat Q$ and two constants $C_1, C_2$ such that
$$\Vert \hat G-G\Vert_{L^\infty(\Omega)}\leq C_1 \Vert \sqrt{G}\Vert^2_{W^\beta_2(\Omega)}\gamma^{1-\frac{D}{2\beta}}, \quad \Vert \hat Q - Q \Vert_{L^\infty(\Omega)}\leq C_2\Vert\sqrt{Q}\Vert^2_{W^\beta_2(\Omega)}\gamma^{1-\frac{D}{2\beta}}$$
where $C_1$ and $C_2$ are independent of $Q,\hat Q, \hat G, G$.

By combining these inequalities with \cref{prop:bound-delta-n}, since $\xi(E) = 1$:
\begin{align}
    \delta_n \leq & \frac{2}{\mixing}\left(C\Vert \sqrt{G}\Vert^2_{W^\beta_2(\Omega)}\gamma^{1- \frac{D}{2\beta}} + C_d \Vert\hat G\Vert_{L^\infty(\Omega)}^2\Vert \sqrt{Q}\Vert^2_{W^\beta_2(\Omega)}\gamma^{1 - \frac{D}{2\beta}}\right).
\end{align}
Since $\Vert \hat G\Vert_{L^\infty(\Omega)} \leq \Vert G - \hat G \Vert_{L^\infty(\Omega)} + 2 \Vert \sqrt{G}\Vert_{L^\infty(\Omega)}^2 \leq \Vert G - \hat G \Vert_{L^\infty(\Omega)} + 2 \Vert \sqrt{G}\Vert^2_{W^\beta_2(\Omega)}$,
\begin{align*}
    \delta_n \leq & \frac{2}{\mixing}\left(C\Vert \sqrt{G}\Vert^2_{W^\beta_2(\Omega)}\gamma^{1- \frac{D}{2\beta}} + \left(C + C\epsilon^{1- \frac{D}{2\beta}}\right)\Vert\sqrt{G}\Vert^2_{W^\beta_2(\Omega)}\Vert\sqrt{Q}\Vert^2_{W^\beta_2(\Omega)}\gamma^{1 - \frac{D}{2\beta}}\right)
\end{align*}
Then, by upper-bounding the negligible terms, there exist two constants $C_1, C_2>0$ independent of $Q, \hat Q, G, \hat G$, such that :
\begin{align}
    \delta_n \leq & \underbrace{C_1\left(C_2 \Vert\sqrt{G}\Vert^2_{W^\beta_2(\Omega)} + C_2\Vert \sqrt{G}\Vert^2_{W^\beta_2(\Omega)}\Vert \sqrt{Q}\Vert^2_{W^\beta_2(\Omega)} \right)}_{C^\prime(Q, G)} \frac{\gamma^{1-\frac{D}{2\beta}}}{\sigma}
\end{align}
Note that $C^\prime$ only depends on parameters on $\Vert \sqrt{G}\Vert_{W^\beta_2(\Omega)}, \Vert \sqrt{Q}\Vert_{W^\beta_2(\Omega)}, d, \beta, \Omega$ and does not depend on $\hat{Q}, \hat{G}, \sigma, \gamma$.

We now apply \cref{theorem:learning}.
Let $\varepsilon > 0$ and $\delta > 0$. As a consequence of the development above, if $\gamma$ is chosen such that $\gamma = \left(\frac{\varepsilon}{C^\prime}\right)^{\frac{2\beta}{2\beta - D}}/\sigma$ then (1) $M, n$ correspond to the ones stated in the statement of the theorem (2)  with probability at least $1-6\delta$,
\begin{align}
    \delta_n \leq \frac{\varepsilon}{\sigma}
\end{align}
and then (3),



% TODO THC 16/07
\section{Computations on Generalized Gaussian PSD Models}\label{sec:proof-ops}
The stability properties of Generalized Gaussian PSD Models under probabilistic operations rely at a high-level on the fact that $Tr(AB)Tr(CD)=Tr(A\otimes C B \otimes D)$ and that if $A$ and $B$ are positive semi-definite matrices then so is $A\otimes B$.

\subsection{\textsc{Integral}}
\begin{proposition}[Integration of a Generalized Gaussian PSD Model]
   Let $f(x) = Tr(AB(x))$ with parameters $\left\lbrace A, C, (P_{ij}), (\mu_{ij})\right\rbrace$. Then, $Z =\int f(x)dx$ where
   \begin{align}
       Z = Tr(A \circ \exp^\circ(C) \circ C(P))
   \end{align}
   where $\exp^\circ$ is the element-wise exponential map and $C(P)\in\mathcal S(\mathbb R^M)$ is decribed by $C(P)_{ij}=C(P_{ij})$.
\end{proposition}
We denote $\circ$ the Hadamard product.
\begin{proof}
The proof is clear by linearity of the trace.
\end{proof}
\begin{remark}[Computational complexity]
Because of the need to compute the determinant of $P_{ij}$ the computational complexity of the partial evaluation operation is $O(M^2 d!)$.
\end{remark}
\subsection{\textsc{PartialEval}}
\begin{proposition}[Partial evaluation of a Generalized Gaussian PSD Model]
   Let $f(x, y) = Tr(AB(x, y))$ with parameters $\left\lbrace A, C, (P_{ij}), (\mu_{ij})\right\rbrace$ and $y\in \mathbb R^d$. Then, $g(x):=f(x, y) = Tr(A^\prime B^\prime(x))$ with parameters $\left\lbrace A^\prime, C^\prime, (P^\prime_{ij}), (\mu^\prime_{ij})\right\rbrace$ where
   \begin{align}
       A^\prime &= A\\
       P^\prime_{ij} &= P_{ijxx}\\
       \mu^\prime_{ij} &= \mu_{ijx} + P_{ijxx}^{-1}P_{ijxy}\left(\mu_{ijy} - y\right)\\
        C_{ij}^\prime &= C_{ij} + \nu P_{xx}^{-1}\nu - yP_{yy}y + 2\mu_{x}P_{xy}y + 2\mu_yP_{yy}y-\mu P\mu
       \end{align}
where \begin{equation}
P = \left(\begin{array}{c c}
    P_{xx} & P_{xy}  \\
    P_{xy}^T & P_{yy}
\end{array} \right).
\end{equation}
\end{proposition}
\begin{proof}
We can compute $B^\prime(x)$ by expanding $C - \log(B(x, y)_{ij})$ for any $i,j$. Dropping the $i,j$ dependence:
\begin{align}
     \left\Vert P^{1/2}\left(\begin{bmatrix}x\\ y\end{bmatrix} - \mu \right)\right\Vert^2 =&  x^TP_{xx}x + 2y P_{xy}^Tx + yP_{yy}y - 2\mu P\begin{bmatrix}x\\ y\end{bmatrix}  + \mu P \mu\\
    =&  x^TP_{xx}x + 2y P_{xy}^Tx + yP_{yy}y - 2\mu_x P_{xx}x -2\mu_xP_{xy}y -2\mu_yP_{xy}x -2\mu_yP_{yy}y + \mu P \mu\\
    =&  x^TP_{xx}x -2(\underbrace{\mu_xP_{xx}+\mu_y P_{xy} -yP_{xy}^T}_\nu)x + yP_{yy}y -2\mu_xP_{xy}y -2\mu_yP_{yy}y + \mu P \mu\\
    =&  \left\Vert P_{xx}^{1/2}\left(x - P_{xx}^{-1}\nu\right)\right\Vert^2 - \nu P_{xx}^{-1}\nu + yP_{yy}y -2\mu_xP_{xy}y -2\mu_yP_{yy}y + \mu P \mu.
\end{align}
\end{proof}
\begin{remark}[Computational complexity]
Because of the need to compute the inverse of $P_{xx}$, the computational complexity of the partial evaluation operation is $O(M^2d_x^3)$.
\end{remark}

\subsection{\textsc{Marginalization}}
\begin{proposition}[Marginalization of a Generalized Gaussian PSD Model]
   Let $f(x, y) = Tr(AB(x, y))$ with parameters $\left\lbrace A, C, (P_{ij}), (\mu_{ij})\right\rbrace$. Then, $h(x):=\int f(x, y)dy = Tr(A^\prime B^\prime(x))$ with parameters $\left\lbrace A^\prime, C^\prime, (P^\prime_{ij}), (\mu^\prime_{ij})\right\rbrace$ where
   \begin{align}
       A^\prime &= A\\
        P_{ij}^\prime &= \left(\left[P^{-1}_{ij}\right]_{xx}\right)^{-1}\\
       \mu^\prime_{ij} &= \left[\mu_{ij}\right]_{x}\\
          C^\prime_{ij} &= C_{ij} + \log(C(P_{ij})) - \log(C(P^ \prime_{ij}))
       \end{align}
\end{proposition}
\begin{proof}
We compute the integral component-wise, denoting $\Sigma = (2P)^{-1}$:

\begin{align}
    \int B(x, y)_{ij}dy &= e^{C_{ij}}\int \exp\left(- \frac{\left\Vert \sqrt{2}P_{ij}^{1/2}\left[\begin{pmatrix}x\\ y\end{pmatrix} - \mu_{ij}\right] \right\Vert^2}{2}\right)dy\\
    &= e^{C_{ij}}\int \exp\left(- \frac{\left\Vert \Sigma^{-1/2}\left[\begin{pmatrix}x\\ y\end{pmatrix} - \mu_{ij}\right] \right\Vert^2}{2}\right)dy\\
    &= e^{C_{ij}}\sqrt{(2\pi)^{d_x + d_y}\vert \Sigma \vert}\int \frac{1}{\sqrt{(2\pi)^{d_x + d_y}\vert \Sigma \vert}}\exp\left(- \frac{\left\Vert \Sigma^{-1/2}\left[\begin{pmatrix}x\\ y\end{pmatrix} - \mu_{ij}\right] \right\Vert^2}{2}\right)dy\\
    &= e^{C_{ij}}\sqrt{(2\pi)^{d_x + d_y}\vert \Sigma\vert}\frac{1}{\sqrt{(2\pi)^{d_x}\vert \left[\Sigma\right]_{xx}\vert}}\exp\left(- \frac{\left\Vert [\Sigma]_{xx}^{-1/2}\left( x - \left[\mu_{ij}\right]_x\right)\right\Vert^2}{2}\right)\\
    &= e^{C_{ij}}\frac{C_{P_{ij}}}{C_{P_{ij}^\prime}}\exp\left(-\left\Vert {P_{ij}^\prime}^{1/2}\left( x - \left[\mu_{ij}\right]_x\right)\right\Vert^2\right)\\
    &= e^{C^\prime_{ij}}\exp\left(-\left\Vert {P_{ij}^\prime}^{1/2}\left( x - \mu_{ij}^\prime\right)\right\Vert^2\right)
\end{align}

where $C^\prime_{ij} = C_{ij} + \log(C_{P_{ij}}) - \log(C_{P^\prime_{ij}})$, $P^\prime_{ij} = \left(\left[P_{ij}^{-1}\right]_{xx}\right)^{-1}$ and $\mu_{ij}^\prime = \left[\mu_{ij}\right]_x$.
\end{proof}

\begin{remark}[Computational complexity]
Because of the need to compute the determinant of $P$ as well as invert it, the computational complexity of the partial evaluation operation is $O(M^2\max(d!, d^3))$.
\end{remark}
\subsection{\textsc{Product}}
\begin{proposition}[Product of two Generalized Gaussian PSD Models]
   Let $f(x, y) = Tr(AB(x, y))$ a generalized PSD model of order $M$ with parameters $\left\lbrace A, C, (P_{ij}), (\mu_{ij})\right\rbrace$.
   Let $g(x, y) = Tr(\tilde A \tilde B(x, y))$ a generalized PSD model of order $m$ with parameters $\left\lbrace \tilde A, \tilde C, (\tilde P_{kl}), (\tilde\mu_{kl})\right\rbrace$.

   Then, $h(x):=f(x)g(x)= Tr(A^\prime B^\prime(x))$ is a generalized PSD model of order $Mm$ with parameters $\left\lbrace A^\prime, C^\prime, (P^\prime_{ij}), (\mu^\prime_{ij})\right\rbrace$ where
    \begin{align}
    A^\prime &= A \otimes \tilde A\\
    P_{ijkl}^\prime &= \left[\begin{array}{ccc}
    P_{ijxx} & P_{ijxy}^T & 0 \\
    P_{ijxy} &P_{ijyy} + \tilde P_{klyy}  & \tilde P_{klyz} \\
    0 & \tilde P_{klyz} & \tilde P_{klzz}
    \end{array}\right]\\
    \mu^\prime_{ijkl} &= {P^\prime_{ijkl}}^{-1}\hat\mu_{ijkl}\\
    C_{ijkl} &= C^f_{ij} + C^g_{kl} +\hat\mu_{ijkl}P_{ijkl}\hat\mu_{ijkl} - \mu_{ij}P_{ij}\mu_{ij} - \tilde\mu_{kl}\tilde P_{kl}\tilde\mu_{kl}
\end{align}

where \begin{equation}
    \hat\mu_{ijkl} =\begin{pmatrix}
    {P_{ij}}\mu_{ij}\\
    0\end{pmatrix} +
    \begin{pmatrix}
     0 \\
    {\tilde P_{kl}}\tilde\mu_{kl}
    \end{pmatrix} =
    \begin{pmatrix}
    [{P_{ij}}\mu_{ij}]_x\\
    [{P_{ij}}\mu_{ij}]_{y} + [{\tilde P_{kl}}\tilde\mu_{kl}]_{y}\\
    [{\tilde P_{kl}}\tilde\mu_{kl}]_{z}
     \end{pmatrix}
     \end{equation}
\end{proposition}

\begin{proof}
Notice that:
    \begin{align}
        \log\left(B_f(x, y) \otimes B_g(y, z)_{ijkl}\right)& = C^f_{ij} + C^g_{kl} - \left\Vert {P_{ij}}^{1/2}\left[\begin{pmatrix}x\\ y\end{pmatrix}- \mu_{ij}\right]\right\Vert^2 - \left\Vert {\tilde P_{kl}}^{1/2}\left[\begin{pmatrix}x \\ y \end{pmatrix}- \tilde\mu_{kl}\right]\right\Vert^2
    \end{align}
Let us compute the following term by computing the square:
    \begin{align}
     &\left\Vert {P_{ij}}^{1/2}\left[\begin{pmatrix}x\\y\end{pmatrix}- \mu_{ij}\right]\right\Vert^2 + \left\Vert {\tilde P_{kl}}^{1/2}\left[\begin{pmatrix}y\\ z\end{pmatrix}- \tilde\mu_{kl}\right]\right\Vert^2 \\
     &= \begin{pmatrix}x\\ y\\ z\end{pmatrix}^TP_{ijkl}\begin{pmatrix}x\\ y\\ z\end{pmatrix}- 2\hat\mu_{ijkl}^T\begin{pmatrix}x\\ y\\ z\end{pmatrix} +\mu_{ij}P_{ij}\mu_{ij} + \tilde\mu_{kl}\tilde P_{kl}\tilde\mu_{kl}\\
     &= \left\Vert P_{ijkl}^{1/2}\left(\begin{pmatrix}x\\ y\\ z\end{pmatrix} - P_{ijkl}^{-1}\hat\mu_{ijkl}\right) \right\Vert^2 - \hat\mu_{ijkl}P_{ijkl}^{-1}\hat\mu_{ijkl}+\mu_{ij}P_{ij}\mu_{ij} + \tilde\mu_{kl}\tilde P_{kl}\tilde\mu_{kl}\\
    \end{align}

    where
    \begin{equation}
    P_{ijkl} = \left[\begin{array}{ ccc }
    {P_{ij}}_{xx} & {P_{ij}}_{xy}^T & 0 \\
    {P_{ij}}_{xy} &P_{ijyy} + \tilde P_{klyy}  & \tilde P_{klyz} \\
    0 & \tilde P_{klyz} & \tilde P_{klzz}
    \end{array}\right],
\end{equation}
    \begin{equation}
    \hat\mu_{ijkl} = \begin{pmatrix}
    {P_{ij}}\mu_{ij}\\
    0\end{pmatrix} +
    \begin{pmatrix}
     0 \\
    {\tilde P_{kl}}\tilde\mu_{kl}
    \end{pmatrix} =
    \begin{pmatrix}
    [{P_{ij}}\mu_{ij}]_x\\
    [{P_{ij}}\mu_{ij}]_{y} + [{\tilde P_{kl}}\tilde\mu_{kl}]_{y}\\
    [{\tilde P_{kl}}\tilde\mu_{kl}]_{z}
     \end{pmatrix}
\end{equation}

So
\begin{equation}
    (B_f(x, y) \otimes B_g(y, z))_{ijkl} =  \exp\left(C_{ijkl}-\left\Vert P_{ijkl}^{1/2}\left((x, y, z) - \mu_{ijkl}\right)\right\Vert^2\right)
\end{equation}
where
\begin{equation}
    \mu_{ijkl} = P_{ijkl}^{-1}\hat\mu_{ijkl}
\end{equation}
\begin{equation}
    C_{ijkl} = C^f_{ij} + C^g_{kl} +\hat\mu_{ijkl}P_{ijkl}^{-1}\hat\mu_{ijkl} -  \mu_{ij}P_{ij}\mu_{ij} - \tilde\mu_{kl}\tilde P_{kl}\tilde\mu_{kl}
\end{equation}
\end{proof}
\begin{remark}[Computational complexity]
Because of the need to compute the inverse of $P^\prime$ the computational complexity of the product operation between models of order $M$ and $m$ is $O(M^2m^2d^3)$.
\end{remark}


\subsection{Proof of \cref{theorem:kalman}}\label{sec:proof_kalman}
\begin{proof}
Let $P = L^T\Sigma^{-1}L$ where $L = (F ~-I)$ and $P_\lambda = P + \lambda I$.

We have

\begin{align}
    -\log p(y|x) &= -C_\Sigma + \Vert \Sigma^{-1/2}(Fx + b -y)\Vert^2 = -C_\Sigma + \Vert \Sigma^{-1/2}(Lu + b)\Vert^2\\
                 &= -C_\Sigma + uL^T\Sigma^{-1}Lu - 2b^T\Sigma^{-1}Lu + b^T\Sigma^{-1}b
\end{align}

If we define $\mu = P_{\lambda}^{-1}\beta$ where $\beta = L^T\Sigma^{-1}b$ then,
\begin{align}
    -\log p(y|x) & = -C + \Vert P_\lambda^{1/2}\left(u - \mu\right)\Vert^2\\
                 & = -C + u^T P_\lambda u - 2 \mu^T P_\lambda u + \mu^T P_\lambda \mu\\
                 & = -C + u^TPu + \lambda \Vert u \Vert^2 -2 \beta^Tu + \beta^TP_\lambda^{-1}P_\lambda P_\lambda^{-1}\beta\\
                 & = -C + u^TPu + \lambda \Vert u \Vert^2 -2 \beta^Tu + \beta^TP_\lambda^{-1}\beta\\
\end{align}

And so,
\begin{align}
    - \log(p(y|x)/\hat p(y/x)) &= -C_\Sigma + uPu - 2\beta^Tu + b^T\Sigma^{-1}b + C - u^TPu - \lambda \Vert u \Vert^2 +2 \beta^Tu - \beta^TP_\lambda^{-1}\beta\\
   &= C -C_\Sigma+ b^T\Sigma^{-1}b - \lambda \Vert u \Vert^2 - \beta^TP_\lambda^{-1}\beta
\end{align}

Using Woodbury,
\begin{align}
    b^T \Sigma^{-1}b - \beta^TP_\lambda^{-1}\beta &= b^T\left(\Sigma^{-1} - \Sigma^{-1}L\left[ L^T\Sigma^{-1}L + \lambda I\right]^{-1}L^T-\Sigma^{-1}\right)b\\
    &=\lambda b^T\left( \lambda \Sigma + LL^T\right)^{-1}b
\end{align}

and

\begin{align}
    - \log(p(y|x)/\hat p(y/x)) &= C -C_\Sigma + \lambda b^T\left( \lambda \Sigma + LL^T\right)^{-1}b- \lambda \Vert u \Vert^2
\end{align}

With $C = C_\Sigma - \lambda b^T\left( \lambda \Sigma + LL^T\right)^{-1}b$, $\frac{p(y|x)}{\hat p(x, y)} = e^{\lambda \Vert u \Vert^2}$ and the result follows.
\end{proof}

\subsection{Learning Generalized Gaussian PSD Models}\label{sec:learning-general}
From an approximation perspective, a Generalized Gaussian PSD Model is a Gaussian PSD Model in which one can optimize the anchor points $\tilde x$ and precision matrices $P$ of each kernel function. In the case of approximating transition kernels, this can yield significant improvements in model order. Indeed, a transition kernel $Q(u, x)$ is a conditional probability distribution which depends in which the probability of the value $x$ depends on the value $u$. This dependence is encoded in the combination of kernel evaluations but not in the kernel evaluations themselves.

To approximate a function $f$ with a Generalized Gaussian PSD Model, we implicitly approximate the square-root of $f$ using a Gaussian Linear Model:
\begin{align}\label{eq:non-convex}
    \min_{\hat g \in \mathcal G_M} \frac{1}{n}\sum_{i=1}^n \left\vert f(x_i)- \hat g(x_i)^2\right\vert^2,
\end{align}

where $x_i$ are sampled or chosen on a grid, and $\mathcal G_M = \lbrace \sum_{j=1}^M\alpha_i k_{P_i}(x, \mu_i) ~\vert~ \mu_i \in \mathbb R^d, P = R_i^\top R_i, R_i \in\mathbb R^{d\times d}, \alpha_i \in \mathbb R \text{ for } 1\leq j\leq M \rbrace$. \cref{eq:non-convex} is a smooth, non-convex problem which can be solved approximately using off-the-shelf solver like L-BFGS \citep{lbfgs}.

In practice, we initialize the model by placing $\mu_i$ is regions where $f(\mu_i)$ is large. In the case where $f(u, x)$ is a transition kernel $Q(u, x)$, one strategy is to chose $[\mu_i]_u$ on a grid (or sampled uniformly) and then choose $[\mu_i]_v$ such that $f(\mu_i) = \sup_x f([\mu_i]_u, v)$. This is particularly interesting when $Q$ is a non-linear Gaussian model $Q(u, x) \propto e^{-\Vert\Sigma^{-1/2}(x - h(u))\Vert^2}$ for some non-linear transition model $h$.
